\newglossaryentry{swipe}
{
    name=Swipe,
    description={Ein Swipe ist eine schnelle Wischbewegung eines (oder 				mehrerer) Fingers über einen Touchscreen um eine Funktion auszuführen}
}

\newglossaryentry{atomareinteraktion}
{
    name=Atomare Interaktion,
    description={Der Begriff Atomare Interaktion wurde von Dan Saffers Konzept 		der Microinteractions abgeleitet und beschreibt eine noch feinere 				Betrachtung einer Nutzerinteraktion},
    plural={Atomare Interaktionen}
}

\newglossaryentry{microinteraction}
{
    name=Microinteraction,
    description={Microinteractions sind die funktionalen, interaktiven Details 		eines jeden Produkts. Eine Microinteraction besteht aus vier Elementen: 		Einem Auslöser („Trigger“), Regeln („Rules“), Feedback und Schleifen und 		Modi („Loops and Modes“)}
}
\newglossaryentry{bump}
{
	name=Bump,
	description={Engl. für zusammenstoßen}
}

\newglossaryentry{synchronegeste}
{
	name=Synchrone Geste,
	description={SYNCHRONE GERSTE TOOLTIP}
}

\newglossaryentry{accelerometer}
{
	name=Accelerometer,
	description={Engl. für Beschleunigungssensor. Misst und liefert Daten 			über die Bewegung eines Geräts}
}

\newglossaryentry{vermittlungskomponente}
{
	name=Vermittlungskomponente,
	description={Ein Server, der von den beteiligten Geräten die relevanten 		Daten einer Geste empfängt, miteinander vergleicht und entscheidet ob die 	Geste \textit{erfolgreich} oder \textit{erfolgreich nicht} ist. Der 			Server kann sich entweder lokal auf einem der Geräte befinden, oder von 		beiden über das Internet angesprochen werden}
}