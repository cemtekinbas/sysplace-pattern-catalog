\documentclass[11pt,a4paper,notitlepage]{article}

\usepackage[utf8]{inputenc}
\usepackage[T1]{fontenc}
\usepackage[german]{babel}
\usepackage{float}
%\usepackage{amsmath}
%\usepackage{amsfonts}
\usepackage{amssymb}
\usepackage{graphicx}
\usepackage{ifthen}
\usepackage{diagbox}
\usepackage[hyphens]{url}
\usepackage{textcomp,gensymb}
\usepackage{makecell}
\usepackage{textpos}
\usepackage{tabularx}
\usepackage{csquotes}
%\usepackage{hyperref}
\usepackage[natbib=true,bibstyle=numeric,backend=bibtexu,citestyle=numeric]{biblatex}
\bibliography{lit.bib} 
\renewcommand\theadalign{cb}
\renewcommand\theadfont{\bfseries}
\renewcommand\theadgape{\Gape[4pt]}

\usepackage{nopageno}

\author{}
\date{}
\title{\name}

% Template für Checkboxen ----
\newcommand{\checkbox}[1]{
\ifx#1\undefined
  $\Box$
\else
  $\boxtimes$  
\fi}

\setlength{\parindent}{0pt}

%----------------------------

\newcommand{\grafischedarstellung}{\jobname_graphical_description.png}

%----------------------------

\newcommand{\umldiagram}{\jobname_uml.png}

%----------------------------

\newcommand{\sequencediagram}{\jobname_sequence.png}

%----------------------------

\newcommand{\solutionimg}{\jobname_solution.png}

%----------------------------

\newcommand{\prototypeimg}{\jobname_prototype.png}

% --------- Glossary
\newcommand{\sen}{Sender}
\newcommand{\rec}{Empfänger}
\newcommand{\recdev}{Empfangsgerät}
\newcommand{\sendev}{Sendegerät}
\newcommand{\data}{Datenobjekt}

\newcommand{\name}{Bump To Exchange}

% -------------------------------
% WAS
% -------------------------------

\newcommand{\desc}{Ein Datenobjekt auf einem Sender-Gerät soll von einem Empfänger-Gerät angefordert werden. Das Datenobjekt soll auch auf dem Empfänger-Gerät verfügbar sein.}

\newcommand{\solution}{Gerät A wird mit Gerät B leicht zusammengestoßen/gebumpt. Dadurch wird ein Datensatz von Gerät A nach Gerät B gesendet.}

%\newcommand{\category}{give}
%\newcommand{\category}{take}
\newcommand{\category}{exchange}
%\newcommand{\category}{extend}
%\newcommand{\category}{connect}

% -------------------------------
% WIE
% -------------------------------

\newcommand{\useraction}{Zwei Benutzer halten jeweils ein Gerät fest in der Hand und lassen die Geräte an der Seitenkante zusammenstoßen/bumpen. }

\newcommand{\reaction}{Die Geräte geben nach dem Zusammenstoß eine visuelle oder akustische Rückmeldung an die Benutzer.\\
Es wird signalisiert, dass ein Bump vom Endgerät festgestellt wurde z.B. über Vibration. \\
Auf dem Quellgerät angezeigte Dokumente werden auf das Zielgerät übertragen und dort angezeigt.}

\newcommand{\reactionSuccessVisual}{}
%\newcommand{\reactionSuccessAcustic}{}
\newcommand{\reactionSuccessSensitive}{}

%\newcommand{\reactionFailureConnection}{}
\newcommand{\reactionFailureConnectionDesc}{...}

%\newcommand{\reactionFailureNoDevice}{}
\newcommand{\reactionFailureNoDeviceDesc}{...}

%\newcommand{\reactionFailureCompatibility}{}
\newcommand{\reactionFailureCompatibilityDesc}{...}

\newcommand{\designnotes}{
\begin{itemize}
\item Es sollte im besten Fall keinen Abstand zwischen den beiden Geräten geben, um die Erkennung des angelegten Gerätes zu gewährleisten.
\item Der Zusammenstoß der Geräte muss so erfolgen, dass eine Erkennung möglich ist, andererseits aber die Geräte nicht beschädigt werden.
\end{itemize}}

% -------------------------------
% WANN
% -------------------------------

\newcommand{\validcontext}{...}

\newcommand{\simultaneously}{}
%\newcommand{\sequentially}{}

\newcommand{\online}{}
%\newcommand{\offline}{}

\newcommand{\private}{}
\newcommand{\semipublic}{}
\newcommand{\public}{}
\newcommand{\stationary}{}
\newcommand{\onthego}{}

%\newcommand{\leanback}{}
\newcommand{\leanforward}{}

%\newcommand{\single}{}
\newcommand{\collaboration}{}
\newcommand{\facetoface}{}
%\newcommand{\sidetoside}{}
%\newcommand{\cornertocorner}{}

\newcommand{\smalltask}{}
%\newcommand{\repeatedtask}{}
%\newcommand{\locationbased}{}
%\newcommand{\distraction}{}
%\newcommand{\urgent}{} 

\newcommand{\notvalidcontext}{--- keine Information ---}


\newcommand{\devicetabular}{
\begin{tabular}[H]{|c|c|c|c|c|c|}
\hline 
\diagbox{von}{nach}   & Smartwatch & Smartphone & Tablet & Tabletop & Screens \\ 
\hline 
Smartwatch            &            &            &        &          &         \\ 
\hline 
Smartphone            &            &     x      &   x    &     x    &         \\ 
\hline 
Tablet                &            &            &   x    &     x    &         \\ 
\hline 
Tabletop              &            &            &        &          &         \\ 
\hline
Screens               &            &            &        &          &         \\ 
\hline 
\end{tabular} }

% -------------------------------
% WARUM
% -------------------------------

\newcommand{\established}{}
%\newcommand{\candidate}{}
%\newcommand{\realizable}{}
%\newcommand{\futuristic}{}

\newcommand{\otherpatterns}{
\begin{itemize}
\item Bump To Give
\item Bump To Exchange
\item Bump To Connect
\item Nudge
\end{itemize}
}

\newcommand{\stateoftheart}{
\begin{enumerate}
\item Bump App [1]: Bis 2014 in den App/Play Stores erhältlich gewesen. 
\end{enumerate}
}


\newcommand{\designprinciples}{}

\newcommand{\imageschemata}{}
\newcommand{\imageSchemaContainer}{}
%\newcommand{\imageSchemaInOut}{}
%\newcommand{\imageSchemaPath}{}
%\newcommand{\imageSchemaSourcePathGoal}{}
%\newcommand{\imageSchemaUpDown}{}
\newcommand{\imageSchemaLeftRight}{}
%\newcommand{\imageSchemaNearFar}{}
%\newcommand{\imageSchemaPartWhole}{}

\newcommand{\realworld}{}
\newcommand{\realworldNaivePhysic}{}
\newcommand{\realworldBodyAwareness}{}
\newcommand{\realworldEnvironmentAwareness}{}
\newcommand{\realworldSocialAwareness}{}

\newcommand{\metaphor}{}
\newcommand{\metaphordesc}{Fistbump, Anstoßen (wie mit Getränken)}

% -------------------------------
% TECHNISCHES
% -------------------------------

\newcommand{\technologyObjectIntimate}{}
\newcommand{\technologyObjectPersonal}{}
%\newcommand{\technologyObjectSocial}{}
%\newcommand{\technologyObjectPublic}{}

\newcommand{\technologyObjectDesc}{-}

\newcommand{\technologyCommunicationServer}{}
\newcommand{\technologyCommunicationAdhoc}{}

\newcommand{\technologyCommunicationDesc}{-}

\newcommand{\technologyOrientationAccelerometer}{}
%\newcommand{\technologyOrientationGPS}{}
%\newcommand{\technologyOrientationGyroskop}{}
%\newcommand{\technologyOrientationAnnaeherung}{}
%\newcommand{\technologyOrientationHoehe}{}
\newcommand{\technologyOrientationBeacons}{}
%\newcommand{\technologyOrientationOther}{}

\newcommand{\technologyOrientationDesc}{-}

\newcommand{\prototype}{Generell können zwei Lösungsansätze unterschieden werden:\\
\begin{itemize}
\item Kommunikation über ein Ad-Hoc Netzwerk, das zwischen den beteiligten Geräten gebildet wird.\\
Vorraussetzungen für Endgeräte:
\begin{itemize}
\item Accelerometer
\item Ad-Hoc-Netzwerktechnologie - Bluetooth/Wi-Fi
\item GPS oder Beacon
\end{itemize}
\item Kommunikation über einen Server zu dem sich beteiligte Geräte verbinden.\\
Vorraussetzungen für Endgeräte:
\begin{itemize}
\item Accelerometer
\item Wi-Fi oder Mobilfunk
\item GPS oder Beacon
\end{itemize}
\end{itemize}
}


% -------------------------------
% SONSTIGES
% -------------------------------

\newcommand{\authors}{Benjamin Grab}
\newcommand{\literature}{
\begin{enumerate}
\item Bump. [Online]. \url{http://bu.mp/}
\item BumpTechnologies. Youtube.com. [Online]. \url{https://www.youtube.com/user/BumpTechnologies}
\item Ken Hinckley. 2003. Synchronous gestures for multiple persons and computers. In Proceedings of the 16th annual ACM symposium on User interface software and technology (UIST '03). ACM, New York, NY, USA, 149-158. DOI=10.1145/964696.964713 
\end{enumerate}
}
\newcommand{\figures}{...}
\newcommand{\versionhistory}{...}
\newcommand{\dateofcreation}{...}
\newcommand{\comments}{...}
\newcommand{\questions}{...}


% template inkludieren --------------

\begin{document}

% ------ fixes the build for all patterns where those new variables haven't been defined yet
\ifdefined\reactionSen
\else
\newcommand{\reactionSen}{tbd.}
\fi

\ifdefined\reactionRec
\else
\newcommand{\reactionRec}{tbd.}
\fi

\ifdefined\microinteractionstabular
\else
\newcommand{\microinteractionstabular}{tbd.}
\fi

\ifdefined\animations
\else
\newcommand{\animations}{tbd.}
\fi

\ifdefined\requiredTechnologies
\else
\newcommand{\requiredTechnologies}{tbd.}
\fi

\ifdefined\implementation
\else
\newcommand{\implementation}{tbd.}
\fi

\maketitle

%----------------------------
% CATEGORY ICON
%----------------------------
\begin{textblock}{2}[0,0](8, -3)
\ifthenelse{\equal{\category}{give}}{\newcommand{\icon}{icon_give.png}}{}
\ifthenelse{\equal{\category}{take}}{\newcommand{\icon}{icon_take.png}}{}
\ifthenelse{\equal{\category}{connect}}{\newcommand{\icon}{icon_connect.png}}{}
\ifthenelse{\equal{\category}{extend}}{\newcommand{\icon}{icon_extend.png}}{}
\ifthenelse{\equal{\category}{exchange}}{\newcommand{\icon}{icon_exchange.png}}{}	
\includegraphics[scale=0.5]{\icon}
\end{textblock}

% -------------------------------
% WAS
% -------------------------------
\section*{Was}

\subsection*{Problem}
\desc

\subsection*{Lösung}
\solution

\subsection*{Grafische Darstellung}
\begin{figure}[H]
\IfFileExists{\jobname_graphical_description.png}{\includegraphics[width=\textwidth]{\grafischedarstellung}}{}
\end{figure}

\subsection*{Kategorie}
\ifthenelse{\equal{\category}{give}}{$\boxtimes$}{$\Box$} Give   |   
\ifthenelse{\equal{\category}{take}}{$\boxtimes$}{$\Box$} Take   |   
\ifthenelse{\equal{\category}{exchange}}{$\boxtimes$}{$\Box$} Exchange   |   
\ifthenelse{\equal{\category}{extend}}{$\boxtimes$}{$\Box$} Extend   |   
\ifthenelse{\equal{\category}{connect}}{$\boxtimes$}{$\Box$} Connect

% -------------------------------
% WIE
% -------------------------------
\newpage
\section*{Wie}

\subsection*{Aktion des \sen s}
\useraction

\subsection*{Reaktionen des \sendev s}
\reactionSen

\subsection*{Reaktionen des \recdev s}
\reactionRec

\subsection*{Übersicht über die Microinteractions}
\microinteractionstabular

\subsubsection*{Animationen}
\animations

\subsection*{Hinweise zur Gestaltung der Interaktion}
\designnotes

% -------------------------------
% WANN
% -------------------------------

\section*{Wann}

\subsection*{Geeigneter Nutzungskontext}
\validcontext

\subsubsection*{Zeit}
\checkbox{\simultaneously} gleichzeitige Nutzung der beteiligten Geräte \\
\checkbox{\sequentially} sequentielle Nutzung der beteiligten Geräte

\subsubsection*{Ort}
\checkbox{\private} privat \\
\checkbox{\semipublic} halb-öffentlich \\
\checkbox{\public} öffentlich \\
\checkbox{\stationary} stationär \\
\checkbox{\onthego} unterwegs 

\subsubsection*{Körperhaltung der Benutzer}
\checkbox{\leanback} Lean-Back \\
\checkbox{\leanforward} Lean-Forward 

\subsubsection*{Teilnehmer}
\checkbox{\single} Einzelnutzer \\
\checkbox{\collaboration} Kollaboration

\subsubsection*{Anordnung zwischen Sender und Empfänger}
\checkbox{\facetoface} Face-To-Face \\
\checkbox{\sidetoside} Side-To-Side

\subsection*{Abzuratender Nutzungskontext}
\notvalidcontext

\subsection*{Geräteklassen}
\devicetabular


% -------------------------------
% WARUM
% -------------------------------

\section*{Warum}
\checkbox{\established} Bewährtes Interaction Pattern \\
\checkbox{\candidate} Interaction Pattern Kandidat: 
\checkbox{\realizable} realisierbar oder
\checkbox{\futuristic} futuristisch

\subsection*{Verwandte Patterns}
\otherpatterns

\subsection*{State of the Art}
\stateoftheart

\subsection*{Checkliste: Entspricht die Interaktion der Definiton einer "Blended Interaction"?}
\checkbox{\designprinciples} Werden die Designprinzipien berücksichtigt?
\begin{itemize}
\item[-] Die Interaktion greift eine Metapher aus der physikalischen Welt auf.
\item[-] Die Interaktion kann in einer Kollaboration ausgeführt werden.
\item[-] Die Interaktion unterstützt einen Workflow/eine Aufgabe.
\item[-] Die Interaktion findet in einer physikalischen Umgebung statt.
\end{itemize} 

\checkbox{\imageschemata} Image Schema/ta liegen zu Grunde.
\begin{itemize}
\item[-] \checkbox{\imageSchemaContainer} Container
\item[-] \checkbox{\imageSchemaInOut} In-Out
\item[-] \checkbox{\imageSchemaPath} Path
\item[-] \checkbox{\imageSchemaSourcePathGoal} Source-Path-Goal
\item[-] \checkbox{\imageSchemaUpDown} Up-Down
\item[-] \checkbox{\imageSchemaLeftRight} Left-Right
\item[-] \checkbox{\imageSchemaNearFar} Near-Far
\item[-] \checkbox{\imageSchemaPartWhole} Part-Whole
\end{itemize}

\checkbox{\realworld} Die real-weltlichen Kenntnisse des Menschen werden berücksichtigt.
\begin{itemize}
\item[-] \checkbox{\realworldNaivePhysic} Naive Physik
\item[-] \checkbox{\realworldBodyAwareness} Body Awareness and Skills
\item[-] \checkbox{\realworldEnvironmentAwareness} Environmental Awareness and Skills
\item[-] \checkbox{\realworldSocialAwareness} Social Awareness and Skills
\end{itemize}

\checkbox{\metaphor} Es ist eine natürliche Interaktion. Metapher/Assoziation: \metaphordesc

% -------------------------------
% TECHNISCHES
% -------------------------------

\section*{Technisches}

\subsection*{Benötigte Technologien}
\requiredTechnologies

\subsection*{Implementierungshinweise}
\implementation

% -------------------------------
% SONSTIGES
% -------------------------------

\section*{Sonstiges}

\subsection*{Autor/en}
\authors

\subsection*{Versionshistorie}
Erstelldatum: \dateofcreation \\
Letzte Änderung am: \versionhistory

\subsection*{Kommentare}
\comments

\subsection*{Offene Fragen}
\questions

\listoffigures

\printbibliography

\clearpage

\printglossaries

\end{document}