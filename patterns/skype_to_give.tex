\documentclass[11pt,a4paper,notitlepage]{article}

\usepackage[utf8]{inputenc}
\usepackage[T1]{fontenc}
\usepackage[german]{babel}
\usepackage{float}
%\usepackage{amsmath}
%\usepackage{amsfonts}
\usepackage{amssymb}
\usepackage{graphicx}
\usepackage{ifthen}
\usepackage{diagbox}
\usepackage[hyphens]{url}
\usepackage{textcomp,gensymb}
\usepackage{makecell}
\usepackage{textpos}
\usepackage{tabularx}
\usepackage{csquotes}
%\usepackage{hyperref}
\usepackage[natbib=true,bibstyle=numeric,backend=bibtexu,citestyle=numeric]{biblatex}
\bibliography{lit.bib} 
\renewcommand\theadalign{cb}
\renewcommand\theadfont{\bfseries}
\renewcommand\theadgape{\Gape[4pt]}

\usepackage{nopageno}

\author{}
\date{}
\title{\name}

% Template für Checkboxen ----
\newcommand{\checkbox}[1]{
\ifx#1\undefined
  $\Box$
\else
  $\boxtimes$  
\fi}

\setlength{\parindent}{0pt}

%----------------------------

\newcommand{\grafischedarstellung}{\jobname_graphical_description.png}

%----------------------------

\newcommand{\umldiagram}{\jobname_uml.png}

%----------------------------

\newcommand{\sequencediagram}{\jobname_sequence.png}

%----------------------------

\newcommand{\solutionimg}{\jobname_solution.png}

%----------------------------

\newcommand{\prototypeimg}{\jobname_prototype.png}

% --------- Glossary
\newcommand{\sen}{Sender}
\newcommand{\rec}{Empfänger}
\newcommand{\recdev}{Empfangsgerät}
\newcommand{\sendev}{Sendegerät}
\newcommand{\data}{Datenobjekt}

\newcommand{\name}{Skype To Give}

% -------------------------------
% WAS
% -------------------------------

\newcommand{\desc}{Ein Benutzer (der \sen) hat ein \data\ und will es mit einem weiteren Benutzer (dem \rec) bzw. einem \recdev\ teilen. Es soll über das Internet vom \sendev\ auf das \recdev\ übertragen werden, sodass es anschließend auf beiden Geräten verfügbar ist.}

\newcommand{\solution}{Der \sen\ hat das \sendev{} in der Hand oder vor sich liegen und führt eine Skype-Konversation. Er nutzt den bequemen Skype File Transfer zur Übertragung des \data{}s.}

\newcommand{\category}{give}
%\newcommand{\category}{take}
%\newcommand{\category}{exchange}
%\newcommand{\category}{extend}
%\newcommand{\category}{connect}

% -------------------------------
% WIE
% -------------------------------

\newcommand{\useraction}{Der \sen\ hält das \sendev\ in der Hand oder hat es vor sich liegen bzw. stehen (z.B. ein Tablet oder Tabletop). Ein \data\ ist entweder explizit vom \sen\ (z.B. ein Foto) oder implizit durch die Applikation (z.B. der aktuelle Bildschirm) zum Versenden ausgewählt worden. \recdev\ und \sendev\ sind bereits durch Skype zum bidirektionalen Datenaustausch verbunden.\\

Der \sen{} wählt durch Rechtsklick auf den \rec{} in seiner Kontaktliste "Datei senden" aus, navigiert im File-Dialog zu einer Datei und klickt auf "senden". Das \data{} wird anschließend auf das \recdev{} übertragen.}

\newcommand{\reactionSen}{Auf dem \sendev{} erscheint eine Fortschrittsanzeige zur Dateiübertragung.}

\newcommand{\reactionRec}{Auf dem \recdev{} erscheint eine Fortschrittsanzeige zur Dateiübertragung. Nach Abschluss der Übertragung kann die Datei bequem im Dateiexplorer angezeigt werden.}

\newcommand{\microinteractionstabular}{}

\newcommand{\animations}{}

\newcommand{\designnotes}{}

% -------------------------------
% WANN
% -------------------------------

\newcommand{\validcontext}{Datenaustausch von Bildern, Videos, Visitenkarten, Social Network IDs, Systembefehlen}

%\newcommand{\simultaneously}{}
\newcommand{\sequentially}{}

\newcommand{\private}{}
\newcommand{\semipublic}{}
\newcommand{\public}{}
\newcommand{\stationary}{}
\newcommand{\onthego}{}

\newcommand{\leanback}{}
\newcommand{\leanforward}{}

\newcommand{\single}{}
\newcommand{\collaboration}{}
\newcommand{\facetoface}{}
%\newcommand{\sidetoside}{}

\newcommand{\notvalidcontext}{- Kritische Daten sollten nicht an (halb-) öffentliche Displays gesendet werden. \\ - Wenn viele drahtlos verbundene Geräte räumlich nah beieinander sind, ist es schwierig zu entscheiden welches der \rec\ ist.}

\newcommand{\devicetabular}{
\begin{figure}[H]
\begin{tabular}{|c|c|c|c|c|c|}
\hline 
\diagbox{von}{nach} & Smartwatch & Smartphone & Tablet & Tabletop & Screens \\ 
\hline 
Smartwatch          &     x      &     x      &   x    &     x    &     x   \\ 
\hline 
Smartphone          &     x      &     x      &   x    &     x    &     x   \\ 
\hline 
Tablet              &     x      &     x      &   x    &     x    &     x   \\ 
\hline 
Tabletop            &            &            &        &     x    &     x   \\ 
\hline
Screens             &            &            &        &          &         \\ 
\hline 
\end{tabular}
\caption{Geräteklassen für das Swipe To Give Pattern} 
\end{figure}
}

% -------------------------------
% WARUM
% -------------------------------

\newcommand{\established}{}
%\newcommand{\candidate}{}
%\newcommand{\realizable}{}
%\newcommand{\futuristic}{}

\newcommand{\otherpatterns}{Stitch, Pinch}

\newcommand{\stateoftheart}{
}

\newcommand{\designprinciples}{}

\newcommand{\imageschemata}{}
%\newcommand{\imageSchemaCenterPeriphery}{}
%\newcommand{\imageSchemaContact}{}
%\newcommand{\imageSchemaFrontBack}{}
%\newcommand{\imageSchemaLocation}{}
%\newcommand{\imageSchemaNearFar}{}
\newcommand{\imageSchemaObject}{}
\newcommand{\imageSchemaPath}{}
\newcommand{\imageSchemaSourcePathGoal}{}
%\newcommand{\imageSchemaScale}{}
%\newcommand{\imageSchemaLeftRight}{}
\newcommand{\imageSchemaContainer}{}
\newcommand{\imageSchemaContent}{}
%\newcommand{\imageSchemaFullEmpty}{}
\newcommand{\imageSchemaInOut}{}
%\newcommand{\imageSchemaSurface}{}
%\newcommand{\imageSchemaMerging}{}
%\newcommand{\imageSchemaSplitting}{}
\newcommand{\imageSchemaMomentum}{}
%\newcommand{\imageSchemaSelfMotion}{}
%\newcommand{\imageSchemaBigSmall}{}
\newcommand{\imageSchemaFastSlow}{}
%\newcommand{\imageSchemaPartWhole}{}

\newcommand{\realworld}{}
\newcommand{\realworldNaivePhysic}{}
\newcommand{\realworldBodyAwareness}{}
\newcommand{\realworldEnvironmentAwareness}{}
\newcommand{\realworldSocialAwareness}{}

\newcommand{\metaphor}{}
\newcommand{\metaphordesc}{Jemandem ein Object durch die Luft zuwerfen.}

% -------------------------------
% TECHNISCHES
% -------------------------------



% -------------------------------
% SONSTIGES
% -------------------------------

\newcommand{\authors}{
Horst Schneider, Hochschule Mannheim}

\newcommand{\versionhistory}{15.02.2017}
\newcommand{\dateofcreation}{15.02.2017}
\newcommand{\comments}{Dieses Pattern ist ein Easteregg.}
\newcommand{\questions}


% template inkludieren --------------

\begin{document}

% ------ fixes the build for all patterns where those new variables haven't been defined yet
\ifdefined\reactionSen
\else
\newcommand{\reactionSen}{tbd.}
\fi

\ifdefined\reactionRec
\else
\newcommand{\reactionRec}{tbd.}
\fi

\ifdefined\microinteractionstabular
\else
\newcommand{\microinteractionstabular}{tbd.}
\fi

\ifdefined\animations
\else
\newcommand{\animations}{tbd.}
\fi

\ifdefined\requiredTechnologies
\else
\newcommand{\requiredTechnologies}{tbd.}
\fi

\ifdefined\implementation
\else
\newcommand{\implementation}{tbd.}
\fi

\maketitle

%----------------------------
% CATEGORY ICON
%----------------------------
\begin{textblock}{2}[0,0](8, -3)
\ifthenelse{\equal{\category}{give}}{\newcommand{\icon}{icon_give.png}}{}
\ifthenelse{\equal{\category}{take}}{\newcommand{\icon}{icon_take.png}}{}
\ifthenelse{\equal{\category}{connect}}{\newcommand{\icon}{icon_connect.png}}{}
\ifthenelse{\equal{\category}{extend}}{\newcommand{\icon}{icon_extend.png}}{}
\ifthenelse{\equal{\category}{exchange}}{\newcommand{\icon}{icon_exchange.png}}{}	
\includegraphics[scale=0.5]{\icon}
\end{textblock}

% -------------------------------
% WAS
% -------------------------------
\section*{Was}

\subsection*{Problem}
\desc

\subsection*{Lösung}
\solution

\subsection*{Grafische Darstellung}
\begin{figure}[H]
\IfFileExists{\jobname_graphical_description.png}{\includegraphics[width=\textwidth]{\grafischedarstellung}}{}
\end{figure}

\subsection*{Kategorie}
\ifthenelse{\equal{\category}{give}}{$\boxtimes$}{$\Box$} Give   |   
\ifthenelse{\equal{\category}{take}}{$\boxtimes$}{$\Box$} Take   |   
\ifthenelse{\equal{\category}{exchange}}{$\boxtimes$}{$\Box$} Exchange   |   
\ifthenelse{\equal{\category}{extend}}{$\boxtimes$}{$\Box$} Extend   |   
\ifthenelse{\equal{\category}{connect}}{$\boxtimes$}{$\Box$} Connect

% -------------------------------
% WIE
% -------------------------------
\newpage
\section*{Wie}

\subsection*{Aktion des \sen s}
\useraction

\subsection*{Reaktionen des \sendev s}
\reactionSen

\subsection*{Reaktionen des \recdev s}
\reactionRec

\subsection*{Übersicht über die Microinteractions}
\microinteractionstabular

\subsubsection*{Animationen}
\animations

\subsection*{Hinweise zur Gestaltung der Interaktion}
\designnotes

% -------------------------------
% WANN
% -------------------------------

\section*{Wann}

\subsection*{Geeigneter Nutzungskontext}
\validcontext

\subsubsection*{Zeit}
\checkbox{\simultaneously} gleichzeitige Nutzung der beteiligten Geräte \\
\checkbox{\sequentially} sequentielle Nutzung der beteiligten Geräte

\subsubsection*{Ort}
\checkbox{\private} privat \\
\checkbox{\semipublic} halb-öffentlich \\
\checkbox{\public} öffentlich \\
\checkbox{\stationary} stationär \\
\checkbox{\onthego} unterwegs 

\subsubsection*{Körperhaltung der Benutzer}
\checkbox{\leanback} Lean-Back \\
\checkbox{\leanforward} Lean-Forward 

\subsubsection*{Teilnehmer}
\checkbox{\single} Einzelnutzer \\
\checkbox{\collaboration} Kollaboration

\subsubsection*{Anordnung zwischen Sender und Empfänger}
\checkbox{\facetoface} Face-To-Face \\
\checkbox{\sidetoside} Side-To-Side

\subsection*{Abzuratender Nutzungskontext}
\notvalidcontext

\subsection*{Geräteklassen}
\devicetabular


% -------------------------------
% WARUM
% -------------------------------

\section*{Warum}
\checkbox{\established} Bewährtes Interaction Pattern \\
\checkbox{\candidate} Interaction Pattern Kandidat: 
\checkbox{\realizable} realisierbar oder
\checkbox{\futuristic} futuristisch

\subsection*{Verwandte Patterns}
\otherpatterns

\subsection*{State of the Art}
\stateoftheart

\subsection*{Checkliste: Entspricht die Interaktion der Definiton einer "Blended Interaction"?}
\checkbox{\designprinciples} Werden die Designprinzipien berücksichtigt?
\begin{itemize}
\item[-] Die Interaktion greift eine Metapher aus der physikalischen Welt auf.
\item[-] Die Interaktion kann in einer Kollaboration ausgeführt werden.
\item[-] Die Interaktion unterstützt einen Workflow/eine Aufgabe.
\item[-] Die Interaktion findet in einer physikalischen Umgebung statt.
\end{itemize} 

\checkbox{\imageschemata} Image Schema/ta liegen zu Grunde.
\begin{itemize}
\item[-] \checkbox{\imageSchemaContainer} Container
\item[-] \checkbox{\imageSchemaInOut} In-Out
\item[-] \checkbox{\imageSchemaPath} Path
\item[-] \checkbox{\imageSchemaSourcePathGoal} Source-Path-Goal
\item[-] \checkbox{\imageSchemaUpDown} Up-Down
\item[-] \checkbox{\imageSchemaLeftRight} Left-Right
\item[-] \checkbox{\imageSchemaNearFar} Near-Far
\item[-] \checkbox{\imageSchemaPartWhole} Part-Whole
\end{itemize}

\checkbox{\realworld} Die real-weltlichen Kenntnisse des Menschen werden berücksichtigt.
\begin{itemize}
\item[-] \checkbox{\realworldNaivePhysic} Naive Physik
\item[-] \checkbox{\realworldBodyAwareness} Body Awareness and Skills
\item[-] \checkbox{\realworldEnvironmentAwareness} Environmental Awareness and Skills
\item[-] \checkbox{\realworldSocialAwareness} Social Awareness and Skills
\end{itemize}

\checkbox{\metaphor} Es ist eine natürliche Interaktion. Metapher/Assoziation: \metaphordesc

% -------------------------------
% TECHNISCHES
% -------------------------------

\section*{Technisches}

\subsection*{Benötigte Technologien}
\requiredTechnologies

\subsection*{Implementierungshinweise}
\implementation

% -------------------------------
% SONSTIGES
% -------------------------------

\section*{Sonstiges}

\subsection*{Autor/en}
\authors

\subsection*{Versionshistorie}
Erstelldatum: \dateofcreation \\
Letzte Änderung am: \versionhistory

\subsection*{Kommentare}
\comments

\subsection*{Offene Fragen}
\questions

\listoffigures

\printbibliography

\clearpage

\printglossaries

\end{document}