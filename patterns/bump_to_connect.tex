\documentclass[11pt,a4paper,notitlepage]{article}

\usepackage[utf8]{inputenc}
\usepackage[T1]{fontenc}
\usepackage[german]{babel}
\usepackage{float}
%\usepackage{amsmath}
%\usepackage{amsfonts}
\usepackage{amssymb}
\usepackage{graphicx}
\usepackage{ifthen}
\usepackage{diagbox}
\usepackage[hyphens]{url}
\usepackage{textcomp,gensymb}
\usepackage{makecell}
\usepackage{textpos}
\usepackage{tabularx}
\usepackage{csquotes}
%\usepackage{hyperref}
\usepackage[natbib=true,bibstyle=numeric,backend=bibtexu,citestyle=numeric]{biblatex}
\bibliography{lit.bib} 
\renewcommand\theadalign{cb}
\renewcommand\theadfont{\bfseries}
\renewcommand\theadgape{\Gape[4pt]}

\usepackage{nopageno}

\author{}
\date{}
\title{\name}

% Template für Checkboxen ----
\newcommand{\checkbox}[1]{
\ifx#1\undefined
  $\Box$
\else
  $\boxtimes$  
\fi}

\setlength{\parindent}{0pt}

%----------------------------

\newcommand{\grafischedarstellung}{\jobname_graphical_description.png}

%----------------------------

\newcommand{\umldiagram}{\jobname_uml.png}

%----------------------------

\newcommand{\sequencediagram}{\jobname_sequence.png}

%----------------------------

\newcommand{\solutionimg}{\jobname_solution.png}

%----------------------------

\newcommand{\prototypeimg}{\jobname_prototype.png}

% --------- Glossary
\newcommand{\sen}{Sender}
\newcommand{\rec}{Empfänger}
\newcommand{\recdev}{Empfangsgerät}
\newcommand{\sendev}{Sendegerät}
\newcommand{\data}{Datenobjekt}

\newcommand{\name}{Bump To Connect}

% -------------------------------
% WAS
% -------------------------------

\newcommand{\desc}{Ein Benutzer (der \sen{}) möchte sein Gerät (das \sendev{}) durch direkten Kontakt mit einem anderen Gerät (dem \recdev{}) verbinden um eine Datenübertragung zu ermöglichen.}

\newcommand{\solution}{Der \sen\ hat das \sendev\ in der Hand. Durch das  zusammenstoßen (\glslink{bump}{Bumpen}) des \sendev s\ mit dem \recdev\ wird der Verbindungsvorgang gestartet.}

%\newcommand{\category}{give}
%\newcommand{\category}{take}
%\newcommand{\category}{exchange}
%\newcommand{\category}{extend}
\newcommand{\category}{connect}

% -------------------------------
% WIE
% -------------------------------

\newcommand{\useraction}{Der \sen\ hält das \sendev\ in der Hand und stößt es leicht an das \recdev\ an. Das \recdev\ kann dabei von einer weiteren Person gehalten werden oder stationär sein (z.B ein Tablet oder Tabletop).}

\newcommand{\reactionSen}{Auf dem \sendev\ sollte zu verschiedenen Phasen des Bumps Feedback gegeben werden. Ein Bump besteht aus den drei \glslink{atomareinteraktion}{atomaren Interaktionen} \textit{Move}, \textit{Abrupt Stop} und \textit{Bump Recognized}.\\
Da es sich bei \textbf{Bump to Connect} um eine \gls{synchronegeste} handelt, werden nach dem Bump die \glslink{accelerometer}{Accelerometerdaten} der beteiligten Geräte verglichen. Vor dem Bump besteht keine direkte Verbindung zwischen den Geräten, daher muss es eine \gls{vermittlungskomponente} geben (z.B. ein externer Server), an die die Daten zur Überprüfung gesendet werden. Sind die Daten ähnlich genug, gilt der Bump als \textit{erfolgreich}. Bei einem \textit{erfolgreich} ausgeführten Bump finden alle atomaren Interaktionen statt, nach Beendigung der letzten werden die Geräte drahtlos verbunden (Atomare Interaktion \textit{Connect}), worüber der Nutzer ebenfalls Feedback erhalten sollte.}

\newcommand{\reactionRec}{Handelt es sich bei dem \recdev\ um ein mobiles Gerät, das von einem Nutzer gehalten wird, werden die selben atomaren Interaktionen ausgeführt wie auf dem \sendev .\
}

\newcommand{\microinteractionstabular}{
\begin{figure}[H]
\begin{table}[H]
\renewcommand{\arraystretch}{2}\addtolength{\tabcolsep}{-2pt}
\centering
\newcolumntype{b}{X}
\newcolumntype{t}{>{\hsize=.3\hsize}X}
\newcolumntype{s}{>{\hsize=.2\hsize}c}
\newcolumntype{m}{>{\hsize=.6\hsize}X}
\begin{tabularx}{\textwidth}{tsbbm}
\thead[X]{Name} & \thead[c]{Typ*} & \thead[X]{Trigger} & \thead[X]{Regeln} & \thead[X]{Feedback} \\
\hline
Move & M & Gerät wurde bewegt (Accelerometerdaten) & \data\ ist ausgewählt &  Animation 1 \\ 
\hline
Abrupt Stop & M & Das Gerät ist mit etwas hartem zusammengestoßen (Accelerometerdaten) & Accelerometerdaten erfüllen Bump-Anforderungen & Animation 2 \\ 
\hline
Bump Recognized & S & Bump-Daten an Vermittlungskomponente gesendet und Antwort erhalten & Accelerometerdaten beider Geräte ähnlich genug & Animation 3 \\ 
\hline
Connect & S & Verbindung zwischen den Geräten wurde hergestellt & Datentransfer \newline ist möglich & Animation 4 \\
\hline
\end{tabularx}
\end{table}
\caption{Atomare Interaktionen für das Bump to Connect Pattern}
\end{figure}
*Typ: (M)anuell, (S)ystem
}

\newcommand{\animations}{
\begin{enumerate}
\item Move-Animation: visualisiert dem Benutzer, dass das physische Bewegen des Geräts eine Funktionalität darstellt (z.B. Ein Objekt auf dem Bildschirm bewegt sich etwas verzögert, als hätte es Masse)
\item Abrupt Stop-Animation: visualisiert dem Benutzer, dass ein Bump auf seinem Gerät erkannt wurde (z.B. Das Objekt bleibt am angestoßenen Rand des Bildschirms)
\item Bump Recognized-Animation: visualisiert dem Benutzer, dass die Bump-Geste richtig erkannt wurde. (z.B. Das Objekt färbt sich grün)
\item Connect-Animation: visualisiert dem Benutzer, dass eine Verbindung hergestellt wurde(z.B. Das Objekt wird zu einem Stecker und bewegt sich in Richtung des anderen Geräts)
\end{enumerate}
}

\newcommand{\designnotes}{
\begin{itemize}
\item[-] Beim Ausführen der Bump-Geste sollten die Geräte direkt aneinandergestoßen werden um eine möglichst gute Erkennung zu gewährleisten. Zudem sollte die Härte des Zusammenstoßes stark genug sein um eine Erkennung zu ermöglichen, jedoch die Geräte nicht beschädigen.
\item[-] Bei jeder Synchronen Connect-Geste muss es eine Vermittlungskomponente geben, an die die relevanten Daten der Geste gesendet werden. Diese vergleicht die empfangenen Daten der beiden involvierten Geräte und gibt positive oder negative Rückmeldung.
\end{itemize}}

% -------------------------------
% WANN
% -------------------------------

\newcommand{\validcontext}{Verbinden von mobilen Geräten zwecks Datenübertragung, Verbinden von mobilen Privatgeräten mit stationären Geräten (z.B. Verbindung zu einem Netzwerk am Arbeitsplatz)}

\newcommand{\simultaneously}{}
%\newcommand{\sequentially}{}

\newcommand{\online}{}
%\newcommand{\offline}{}

\newcommand{\private}{}
\newcommand{\semipublic}{}
\newcommand{\public}{}
\newcommand{\stationary}{}
\newcommand{\onthego}{}

\newcommand{\leanback}{}
\newcommand{\leanforward}{}

\newcommand{\single}{}
\newcommand{\collaboration}{}
\newcommand{\facetoface}{}
\newcommand{\sidetoside}{}
\newcommand{\cornertocorner}{}

\newcommand{\notvalidcontext}{Sichtbarmachen vertraulicher Informationen (z.B. Name, Alter etc.) auf öffentlichen Displays.}


\newcommand{\devicetabular}{
\begin{tabular}[H]{|c|c|c|c|c|c|}
\hline 
\diagbox{von}{nach}   & Smartwatch & Smartphone & Tablet & Tabletop & Screens \\ 
\hline 
Smartwatch            &     x      &     x      &   x    &    x     &         \\ 
\hline 
Smartphone            &     x      &     x      &   x    &     x    &        \\ 
\hline 
Tablet                &     x      &     x      &   x    &     x    &    
\\ 
\hline 
Tabletop              &            &            &        &          &         \\ 
\hline
Screens               &            &            &        &          &         \\ 
\hline 
\end{tabular}}

% -------------------------------
% WARUM
% -------------------------------

%\newcommand{\established}{}
\newcommand{\candidate}{}
\newcommand{\realizable}{}
%\newcommand{\futuristic}{}

\newcommand{\otherpatterns}{
\begin{itemize}
\item Bump To Give
\item Bump To Take
\item Bump To Exchange
\item Nudge
\end{itemize}
}

\newcommand{\stateoftheart}{
\begin{enumerate}
\item Bump App: Bis 2014 in den App/Play Stores erhältlich gewesen [\url{http://bu.mp/}]
\item Beispiel-Implementierung eines Bump Patterns \citep{Grab2015}.
\item Hinckley, K. \citep{Hinckley2003}
%\item Microinteractions im Multi-Screen Kontext \citep{Madden2016}.
\end{enumerate}
}

\newcommand{\designprinciples}{}

\newcommand{\imageschemata}{}
%\newcommand{\imageSchemaVoid}{}
%\newcommand{\imageSchemaObject}{}
%\newcommand{\imageSchemaSubstance}{}
%\newcommand{\imageSchemaCenterPeriphery}{}
\newcommand{\imageSchemaContact}{}
%\newcommand{\imageSchemaFrontBack}{}
%\newcommand{\imageSchemaLocation}{}
%\newcommand{\imageSchemaNearFar}{}
%\newcommand{\imageSchemaPath}{}
%\newcommand{\imageSchemaSourcePathGoal}{}
%\newcommand{\imageSchemaScale}{}
%\newcommand{\imageSchemaLeftRight}{}
%\newcommand{\imageSchemaContainer}{}
%\newcommand{\imageSchemaContent}{}
%\newcommand{\imageSchemaFullEmpty}{}
%\newcommand{\imageSchemaInOut}{}
%\newcommand{\imageSchemaSurface}{}
%\newcommand{\imageSchemaMerging}{}
%\newcommand{\imageSchemaSplitting}{}
\newcommand{\imageSchemaMomentum}{}
%\newcommand{\imageSchemaSelfMotion}{}
%\newcommand{\imageSchemaBigSmall}{}
%\newcommand{\imageSchemaFastSlow}{}
%\newcommand{\imageSchemaPartWhole}{}

\newcommand{\realworld}{}
\newcommand{\realworldNaivePhysic}{}
\newcommand{\realworldBodyAwareness}{}
\newcommand{\realworldEnvironmentAwareness}{}
\newcommand{\realworldSocialAwareness}{}

\newcommand{\metaphor}{}
\newcommand{\metaphordesc}{Fistbump, Anstoßen (wie mit Getränken)}

% -------------------------------
% TECHNISCHES
% -------------------------------

\newcommand{\technologyObjectIntimate}{}
\newcommand{\technologyObjectPersonal}{}
%\newcommand{\technologyObjectSocial}{}
%\newcommand{\technologyObjectPublic}{}

\newcommand{\technologyObjectDesc}{Über die Major- und Minor-Werte kann mit iBeacon jedem Endgerät eine einzigartige ID zugeordnet werden mit der Endgeräte identifiziert werden können. Um diese ID zwischen Bump-Partnern auszutauschen, wird auf beiden Geräten iBeacon zum Zeitpunk des Bumps für kurze Zeit aktiviert. Gerade lange genug, damit die Endgeräte alle aktiven Beacons in ihrem Umfeld sehen können. Dadurch besitzt jedes Gerät eine Liste an Beacons die zu einen bestimmten Zeitpunkt an einem Bump, in ihrer Empfangsreichweite, beteiligt waren. Können die Endgeräte jeweils nur ein anderes Beacon sehen, haben Sie ihren Bump-Partner identifiziert. Ist mehr als ein Beacon sichtbar, fanden mehrere Bumps zeitgleich statt. In diesen Fällen können die Partner, über die Entfernung der Geräte zueinander, ermittelt werden. Bei den Geräten mit dem geringsten Abstand handelt es sich um die Bump-Partner. Die Erkennung der Partnergeräte über die Entfernung macht es erforderlich, dass zeitgleiche Bumps mindestens einige Zentimeter voneinander entfernt stattfinden. Dies stellt sicher, dass eine falsche Zuordnung durch ungenaue Abstandsmessungen vermieden wird.}

%\newcommand{\technologyCommunicationServer}{}
\newcommand{\technologyCommunicationAdhoc}{}

\newcommand{\technologyCommunicationDesc}{Wird auf einem Gerät ein Bump registriert, ist der erste Schritt die Generierung von Zufallszahlen für den Major- und Minor-Wert von iBeacon. Diese Zahlen bilden eine eindeutige ID, mit der sich jedes Gerät im Netzwerk identifizieren kann. Anschließend wird iBeacon aktiviert, die Geräte können sich gegenseitig sehen, GeräteIDs lesen und die Distanz zu allen sichtbaren Beacons erfassen. Ist mehr als ein Beacon sichtbar, wird jenes ermittelt, welches die geringste Distanz zum suchenden Gerät aufweist. Die GeräteID dieses Geräts wird lokal gespeichert und iBeacon wird deaktiviert. Anschließend startet die DiscoveryPhase des Multipeer-Connectivity-Frameworks.}

%\newcommand{\technologyOrientationAccelerometer}{}
%\newcommand{\technologyOrientationGPS}{}
%\newcommand{\technologyOrientationGyroskop}{}
%\newcommand{\technologyOrientationAnnaeherung}{}
%\newcommand{\technologyOrientationHoehe}{}
%\newcommand{\technologyOrientationBeacons}{}
%\newcommand{\technologyOrientationOther}{}

\newcommand{\technologyOrientationDesc}{-}

\newcommand{\prototype}{...}


% -------------------------------
% SONSTIGES
% -------------------------------

\newcommand{\authors}{
Benjamin Grab, Hochschule Mannheim\\
Valentina Burjan, Hochschule Mannheim}
\newcommand{\literature}{
\begin{enumerate}
\item Bump. [Online]. \url{http://bu.mp/}
\item BumpTechnologies. Youtube.com. [Online]. \url{https://www.youtube.com/user/BumpTechnologies}
\item Ken Hinckley. 2003. Synchronous gestures for multiple persons and computers. In Proceedings of the 16th annual ACM symposium on User interface software and technology (UIST '03). ACM, New York, NY, USA, 149-158. DOI=10.1145/964696.964713 
\item \url{https://github.com/informatik-mannheim/thesis-bump-to-transfer/tree/master/sources/
Bumper}\\
Swift-Code zu der Demonstrator Applikationen von Bump
\end{enumerate}
}
\newcommand{\figures}{...}
\newcommand{\versionhistory}{...}
\newcommand{\dateofcreation}{...}
\newcommand{\comments}{...}
\newcommand{\questions}{...}


% template inkludieren --------------

\begin{document}

% ------ fixes the build for all patterns where those new variables haven't been defined yet
\ifdefined\reactionSen
\else
\newcommand{\reactionSen}{tbd.}
\fi

\ifdefined\reactionRec
\else
\newcommand{\reactionRec}{tbd.}
\fi

\ifdefined\microinteractionstabular
\else
\newcommand{\microinteractionstabular}{tbd.}
\fi

\ifdefined\animations
\else
\newcommand{\animations}{tbd.}
\fi

\ifdefined\requiredTechnologies
\else
\newcommand{\requiredTechnologies}{tbd.}
\fi

\ifdefined\implementation
\else
\newcommand{\implementation}{tbd.}
\fi

\maketitle

%----------------------------
% CATEGORY ICON
%----------------------------
\begin{textblock}{2}[0,0](8, -3)
\ifthenelse{\equal{\category}{give}}{\newcommand{\icon}{icon_give.png}}{}
\ifthenelse{\equal{\category}{take}}{\newcommand{\icon}{icon_take.png}}{}
\ifthenelse{\equal{\category}{connect}}{\newcommand{\icon}{icon_connect.png}}{}
\ifthenelse{\equal{\category}{extend}}{\newcommand{\icon}{icon_extend.png}}{}
\ifthenelse{\equal{\category}{exchange}}{\newcommand{\icon}{icon_exchange.png}}{}	
\includegraphics[scale=0.5]{\icon}
\end{textblock}

% -------------------------------
% WAS
% -------------------------------
\section*{Was}

\subsection*{Problem}
\desc

\subsection*{Lösung}
\solution

\subsection*{Grafische Darstellung}
\begin{figure}[H]
\IfFileExists{\jobname_graphical_description.png}{\includegraphics[width=\textwidth]{\grafischedarstellung}}{}
\end{figure}

\subsection*{Kategorie}
\ifthenelse{\equal{\category}{give}}{$\boxtimes$}{$\Box$} Give   |   
\ifthenelse{\equal{\category}{take}}{$\boxtimes$}{$\Box$} Take   |   
\ifthenelse{\equal{\category}{exchange}}{$\boxtimes$}{$\Box$} Exchange   |   
\ifthenelse{\equal{\category}{extend}}{$\boxtimes$}{$\Box$} Extend   |   
\ifthenelse{\equal{\category}{connect}}{$\boxtimes$}{$\Box$} Connect

% -------------------------------
% WIE
% -------------------------------
\newpage
\section*{Wie}

\subsection*{Aktion des \sen s}
\useraction

\subsection*{Reaktionen des \sendev s}
\reactionSen

\subsection*{Reaktionen des \recdev s}
\reactionRec

\subsection*{Übersicht über die Microinteractions}
\microinteractionstabular

\subsubsection*{Animationen}
\animations

\subsection*{Hinweise zur Gestaltung der Interaktion}
\designnotes

% -------------------------------
% WANN
% -------------------------------

\section*{Wann}

\subsection*{Geeigneter Nutzungskontext}
\validcontext

\subsubsection*{Zeit}
\checkbox{\simultaneously} gleichzeitige Nutzung der beteiligten Geräte \\
\checkbox{\sequentially} sequentielle Nutzung der beteiligten Geräte

\subsubsection*{Ort}
\checkbox{\private} privat \\
\checkbox{\semipublic} halb-öffentlich \\
\checkbox{\public} öffentlich \\
\checkbox{\stationary} stationär \\
\checkbox{\onthego} unterwegs 

\subsubsection*{Körperhaltung der Benutzer}
\checkbox{\leanback} Lean-Back \\
\checkbox{\leanforward} Lean-Forward 

\subsubsection*{Teilnehmer}
\checkbox{\single} Einzelnutzer \\
\checkbox{\collaboration} Kollaboration

\subsubsection*{Anordnung zwischen Sender und Empfänger}
\checkbox{\facetoface} Face-To-Face \\
\checkbox{\sidetoside} Side-To-Side

\subsection*{Abzuratender Nutzungskontext}
\notvalidcontext

\subsection*{Geräteklassen}
\devicetabular


% -------------------------------
% WARUM
% -------------------------------

\section*{Warum}
\checkbox{\established} Bewährtes Interaction Pattern \\
\checkbox{\candidate} Interaction Pattern Kandidat: 
\checkbox{\realizable} realisierbar oder
\checkbox{\futuristic} futuristisch

\subsection*{Verwandte Patterns}
\otherpatterns

\subsection*{State of the Art}
\stateoftheart

\subsection*{Checkliste: Entspricht die Interaktion der Definiton einer "Blended Interaction"?}
\checkbox{\designprinciples} Werden die Designprinzipien berücksichtigt?
\begin{itemize}
\item[-] Die Interaktion greift eine Metapher aus der physikalischen Welt auf.
\item[-] Die Interaktion kann in einer Kollaboration ausgeführt werden.
\item[-] Die Interaktion unterstützt einen Workflow/eine Aufgabe.
\item[-] Die Interaktion findet in einer physikalischen Umgebung statt.
\end{itemize} 

\checkbox{\imageschemata} Image Schema/ta liegen zu Grunde.
\begin{itemize}
\item[-] \checkbox{\imageSchemaContainer} Container
\item[-] \checkbox{\imageSchemaInOut} In-Out
\item[-] \checkbox{\imageSchemaPath} Path
\item[-] \checkbox{\imageSchemaSourcePathGoal} Source-Path-Goal
\item[-] \checkbox{\imageSchemaUpDown} Up-Down
\item[-] \checkbox{\imageSchemaLeftRight} Left-Right
\item[-] \checkbox{\imageSchemaNearFar} Near-Far
\item[-] \checkbox{\imageSchemaPartWhole} Part-Whole
\end{itemize}

\checkbox{\realworld} Die real-weltlichen Kenntnisse des Menschen werden berücksichtigt.
\begin{itemize}
\item[-] \checkbox{\realworldNaivePhysic} Naive Physik
\item[-] \checkbox{\realworldBodyAwareness} Body Awareness and Skills
\item[-] \checkbox{\realworldEnvironmentAwareness} Environmental Awareness and Skills
\item[-] \checkbox{\realworldSocialAwareness} Social Awareness and Skills
\end{itemize}

\checkbox{\metaphor} Es ist eine natürliche Interaktion. Metapher/Assoziation: \metaphordesc

% -------------------------------
% TECHNISCHES
% -------------------------------

\section*{Technisches}

\subsection*{Benötigte Technologien}
\requiredTechnologies

\subsection*{Implementierungshinweise}
\implementation

% -------------------------------
% SONSTIGES
% -------------------------------

\section*{Sonstiges}

\subsection*{Autor/en}
\authors

\subsection*{Versionshistorie}
Erstelldatum: \dateofcreation \\
Letzte Änderung am: \versionhistory

\subsection*{Kommentare}
\comments

\subsection*{Offene Fragen}
\questions

\listoffigures

\printbibliography

\clearpage

\printglossaries

\end{document}
