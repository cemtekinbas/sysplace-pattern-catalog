\begin{document}

\maketitle

% -------------------------------
% WAS
% -------------------------------
\section*{Was}

\subsection*{Problem}
\desc

\subsection*{Lösung}
\solution

\subsection*{Grafische Darstellung}
\begin{figure}[H]
\IfFileExists{\jobname_graphical_description.png}{\includegraphics[width=\textwidth]{\grafischedarstellung}}{}
\end{figure}

\subsection*{Kategorie}
\ifthenelse{\equal{\category}{give}}{$\boxtimes$}{$\Box$} Give   |   
\ifthenelse{\equal{\category}{take}}{$\boxtimes$}{$\Box$} Take   |   
\ifthenelse{\equal{\category}{exchange}}{$\boxtimes$}{$\Box$} Exchange   |   
\ifthenelse{\equal{\category}{extend}}{$\boxtimes$}{$\Box$} Extend   |   
\ifthenelse{\equal{\category}{connect}}{$\boxtimes$}{$\Box$} Connect

% -------------------------------
% WIE
% -------------------------------

\section*{Wie}

\subsection*{Aktion des Benutzers}
\useraction

\subsection*{Reaktion des Sende-und Empfänger-Gerätes}
\reaction

\subsubsection*{Reaktion im Erfolgsfall:}
\checkbox{\reactionSuccessVisual} visuelle Rückmeldung \\
\checkbox{\reactionSuccessAcustic} akustische Rückmeldung, z.B. Ton \\
\checkbox{\reactionSuccessSensitive} sensitive Rückmeldung, z.B. Vibration

\subsubsection*{Reaktion im Nicht-Erfolgsfall:}
\checkbox{\reactionFailureConnection} wenn keine funktionierende Verbindung besteht \reactionFailureConnectionDesc \\
\checkbox{\reactionFailureNoDevice} wenn das Zielgerät nicht erkannt wurde \reactionFailureNoDeviceDesc \\
\checkbox{\reactionFailureCompatibility} wenn das Zielgerät nicht kompatibel ist \reactionFailureCompatibilityDesc

\subsection*{Hinweise zur Gestaltung der Interaktion}
\designnotes

% -------------------------------
% WANN
% -------------------------------

\section*{Wann}

\subsection*{Geeigneter Nutzungskontext}
\validcontext

\subsubsection*{Zeit}
\checkbox{\simultaneously} gleichzeitige Nutzung von Geräten \\
\checkbox{\sequentially} sequentielle Nutzung von Geräten 

\subsubsection*{Modus}
\checkbox{\online} online \\
\checkbox{\offline} offline 

\subsubsection*{Ort}
\checkbox{\private} privat \\
\checkbox{\semipublic} halb-öffentlich \\
\checkbox{\public} öffentlich \\
\checkbox{\stationary} stationär \\
\checkbox{\onthego} unterwegs 

\subsubsection*{Position}
\checkbox{\leanback} Lean-Back \\
\checkbox{\leanforward} Lean-Forward 

\subsubsection*{Teilnehmer}
\checkbox{\single} Einzelnutzer \\
\checkbox{\collaboration} Kollaboration

\subsubsection*{Anordnung zwischen Sender und Empfänger}
\checkbox{\facetoface} Face-To-Face \\
\checkbox{\sidetoside} Side-To-Side \\
\checkbox{\cornertocorner} Corner-To-Corner

\subsubsection*{Tätigkeit}
\checkbox{\smalltask} kleinere Aufgabe \\
\checkbox{\repeatedtask} wiederholte Tätigkeit \\
\checkbox{\locationbased} ortsbezogene Informationsbeschaffung \\
\checkbox{\distraction} Ablenkung \\
\checkbox{\urgent} dringendes 

\subsection*{Abzuratender Nutzungskontext}
\notvalidcontext

\subsection*{Geräteklassen}
\devicetabular

% -------------------------------
% WARUM
% -------------------------------

\section*{Warum}

\checkbox{\established} Bewährtes Interaction Pattern \\
\checkbox{\candidate} Interaction Pattern Kandidat: 
\checkbox{\realizable} realisierbar oder
\checkbox{\futuristic} futuristisch 

\subsection*{Analoge Patterns}
\otherpatterns

\subsection*{State of the Art/Gebrauchshistorie}
\stateoftheart

\subsection*{Checkliste: Entspricht die Interaktion der Definiton eines "Blended Interaction"?}
\checkbox{\designprinciples} Werden die Designprinzipien berücksichtigt?
\begin{itemize}
\item[-] Die Interaktion findet in Kombination mit physikalischen Gegenständen statt.
\item[-] Die Interaktion kann in einer Kollaboration ausgeführt werden.
\item[-] Die Interaktion unterstützt einen Workflow/eine Aufgabe.
\item[-] Die Interaktion findet in einer physikalischen Umgebung statt.
\end{itemize} 

\checkbox{\imageschemata} Image Schema/ta liegen zu Grunde.
\begin{itemize}
\item[-] \checkbox{\imageSchemaContainer} Container
\item[-] \checkbox{\imageSchemaInOut} In-Out
\item[-] \checkbox{\imageSchemaPath} Path
\item[-] \checkbox{\imageSchemaSourcePathGoal} Source-Path-Goal
\item[-] \checkbox{\imageSchemaUpDown} Up-Down
\item[-] \checkbox{\imageSchemaLeftRight} Left-Right
\item[-] \checkbox{\imageSchemaNearFar} Near-Far
\item[-] \checkbox{\imageSchemaPartWhole} Part-Whole
\end{itemize}

\checkbox{\realworld} Die real-weltlichen Kenntnisse des Menschen werden berücksichtigt.
\begin{itemize}
\item[-] \checkbox{\realworldNaivePhysic} Naive Physik
\item[-] \checkbox{\realworldBodyAwareness} Body Awareness and Skills
\item[-] \checkbox{\realworldEnvironmentAwareness} Environment Awareness and Skills
\item[-] \checkbox{\realworldSocialAwareness} Social Awareness and Skills
\end{itemize}

\checkbox{\metaphor} Es ist eine natürliche Interaktion. Metapher/Assoziation: \metaphordesc

% -------------------------------
% TECHNISCHES
% -------------------------------

\section*{Technisches}

\subsection*{Technologien zur Objekterkennung}
\checkbox{\technologyObjectIntimate} 0cm bis 50cm, z.B. NFC, Tags, RFID ... \\
\checkbox{\technologyObjectPersonal} 0,5cm bis 1m, z.B. Bluetooth \\
\checkbox{\technologyObjectSocial} 1m bis 4m, z.B. Kamera \\
\checkbox{\technologyObjectPublic} über 4m, z.B. GPS \\

\technologyObjectDesc

\subsection*{Technologien zur Kommunikation}
\checkbox{\technologyCommunicationServer} Server-basiert \\
\checkbox{\technologyCommunicationAdhoc} Ad-hoc-Netzwerk basiert\\

\technologyCommunicationDesc

\subsection*{Technologien zur Bewegungs-/Orientierungsbestimmung}
\checkbox{\technologyOrientationAccelerometer} Accelerometer \\
\checkbox{\technologyOrientationGPS} GPS \\
\checkbox{\technologyOrientationGyroskop} Gyroskop \\
\checkbox{\technologyOrientationAnnaeherung} Annäherungssensor \\
\checkbox{\technologyOrientationHoehe} Höhenmesser \\
\checkbox{\technologyOrientationBeacons} Beacons \\
\checkbox{\technologyOrientationOther} andere \\

\technologyOrientationDesc 

\subsection*{Prototyp/Lösungsansatz/Code-Snippets/UML-Diagramm}
\prototype

\begin{figure}[H]
\IfFileExists{\jobname_sequence.png}{\includegraphics[width=\textwidth]{\sequencediagram}}{}
\end{figure}

\begin{figure}[H]
\IfFileExists{\jobname_solution.png}{\includegraphics[width=\textwidth]{\solutionimg}}{}
\end{figure}

\begin{figure}[H]
\IfFileExists{\jobname_uml.png}{\includegraphics[width=\textwidth]{\umldiagram}}{}
\end{figure}

\begin{figure}[H]
\IfFileExists{\jobname_prototype.png}{\includegraphics[width=\textwidth]{\prototypeimg}}{}
\end{figure}

% -------------------------------
% SONSTIGES
% -------------------------------

\section*{Sonstiges}

\subsection*{Autor/en}
\authors

\subsection*{Literaturreferenzen}
\literature
\nocite{*}
\bibliography{lit.bib}

\subsection*{Abbildungsverzeichnis}
\figures

\subsection*{Versionshistorie}
Letzte Änderung am: \versionhistory \\
Erstelldatum: \dateofcreation

\subsection*{Kommentare}
\comments

\subsection*{Offene Fragen}
\questions

\end{document}