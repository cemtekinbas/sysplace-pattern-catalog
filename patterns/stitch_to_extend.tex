\documentclass[11pt,a4paper,notitlepage]{article}

\usepackage[utf8]{inputenc}
\usepackage[T1]{fontenc}
\usepackage[german]{babel}
\usepackage{float}
%\usepackage{amsmath}
%\usepackage{amsfonts}
\usepackage{amssymb}
\usepackage{graphicx}
\usepackage{ifthen}
\usepackage{diagbox}
\usepackage[hyphens]{url}
\usepackage{textcomp,gensymb}
\usepackage{makecell}
\usepackage{textpos}
\usepackage{tabularx}
\usepackage{csquotes}
%\usepackage{hyperref}
\usepackage[natbib=true,bibstyle=numeric,backend=bibtexu,citestyle=numeric]{biblatex}
\bibliography{lit.bib} 
\renewcommand\theadalign{cb}
\renewcommand\theadfont{\bfseries}
\renewcommand\theadgape{\Gape[4pt]}

\usepackage{nopageno}

\author{}
\date{}
\title{\name}

% Template für Checkboxen ----
\newcommand{\checkbox}[1]{
\ifx#1\undefined
  $\Box$
\else
  $\boxtimes$  
\fi}

\setlength{\parindent}{0pt}

%----------------------------

\newcommand{\grafischedarstellung}{\jobname_graphical_description.png}

%----------------------------

\newcommand{\umldiagram}{\jobname_uml.png}

%----------------------------

\newcommand{\sequencediagram}{\jobname_sequence.png}

%----------------------------

\newcommand{\solutionimg}{\jobname_solution.png}

%----------------------------

\newcommand{\prototypeimg}{\jobname_prototype.png}

% --------- Glossary
\newcommand{\sen}{Sender}
\newcommand{\rec}{Empfänger}
\newcommand{\recdev}{Empfangsgerät}
\newcommand{\sendev}{Sendegerät}
\newcommand{\data}{Datenobjekt}

\newcommand{\name}{Stitch To Extend}

% -------------------------------
% WAS
% -------------------------------

\newcommand{\desc}{Der Bildschirm-Inhalt des Quell-Gerätes soll auf dem Ziel-Gerät erweitert dargestellt werden.}

\newcommand{\solution}{Zwei Geräte werden nebeneinander gelegt. Der Anwender zieht mit dem Finger eine Linie von dem Display des Quell-Gerätes hinüber auf das Display des Ziel-Gerätes, wo die gezogene Linie endet.
Diese Nutzerinteraktion löst die Erweiterung des Bildschirminhalts vom Quell- auf das Ziel-Gerät aus.}

%\newcommand{\category}{give}
%\newcommand{\category}{take}
%\newcommand{\category}{exchange}
\newcommand{\category}{extend}
%\newcommand{\category}{connect}

% -------------------------------
% WIE
% -------------------------------

\newcommand{\useraction}{Der Nutzer legt das Quell- und Zielgerät parallel zueinander Seite an Seite. Auf dem Quell-Gerät wählt er die zu erweiternde Ansicht (z.B. Bild) aus. Er platziert einen Finger auf dem Bildschirm des Quellgerätes und führt diesen auf den Bildschirm des Zielgerätes hinüber, ohne den Finger zu heben.}

\newcommand{\reaction}{Die Bildschirm-Ansicht des Quell-Gerätes wird bei korrekter Ausführung und Erkennung auf dem Ziel-Gerät erweitert, also anteilig, dargestellt. }

\newcommand{\designnotes}{Bei zu unterschiedlichen Displaygrößen der beteiligten Geräte kann ggf. kein Mehrwert erzeugt werden (z.B. wenn ein Tablet und ein Smartphone hochkant nebeneinandergelegt werden kann bei der Erweiterung Fläche verloren gehen (siehe Abb.)).\\
Um einen möglichst fließenden Übergang zu gewinnen müssen die Displays sehr nah und parallel zueinander anliegen.\\
Jeder Bildschirm hat einen Rand, was dazu führt, dass durch das Nebeneinanderlegen zweier Geräte eine Lücke zwischen den Displays entsteht. Dadurch, dass mobile Endgeräte viel kleiner sind als Computermonitore, ist die Lücke zwischen den Geräten relativ zu ihren Dimensionen meist größer.\\
Es gibt zwei Möglichkeiten Ansichten zu erweitern: 
\begin{enumerate}
\item Das Bild in zwei Teile schneiden und die Ränder der Displays ignorieren. Dabei entsteht eine Streckung des Bildes (siehe Abb. /links).
\item Das Bild in zwei Teile schneiden und die Ränder der Displays einkalkulieren und aus dem Bild ausschneiden (siehe Abb. /rechts).
\end{enumerate} }

\newcommand{\microinteractions}{
\begin{enumerate}
\item Test
\item Another test
\end{enumerate}}

\newcommand{\microinteractionstabular}{
\begin{tabular}[H]{|c|c|c|c|c|c|}
\hline 
N\degree   & Name & Trigger Type & Trigger & Rules & Feedback \\ 
\hline 
1            &     Test      &     asdasf       & asdf       &      hhdfsg    &  filoj  \\ 
\hline
2            &    Another Test    &    sad x      &   x  hh  &     (sdfx)* &    asdf    \\ 
\hline 
\end{tabular}}

% -------------------------------
% WANN
% -------------------------------

\newcommand{\validcontext}{
\begin{itemize}
\item Visuelle Erweiterung/Bildschirmerweiterung: z.B. für Foto- / Video- / Spieleanwendungen / Aufteilen einer Arbeitsfläche in zwei / Panorama-Ansicht bei Bildern (siehe Abb.) / Darstellung von Medien im Originalformat (z.B. aufgeschlagenes Buch, also zwei Seiten auf zwei Bildschirmen darstellen)
\item Logische Erweiterung: Auslagern von Programmlogik, um auf den Screens verschiedene Aufgaben durchzuführen, z.B. Fernsehen und auf dem zweiten Screen ein dazugehörige Aktion, wie ein Quiz auszuführen.
\end{itemize}
}

\newcommand{\simultaneously}{}
%\newcommand{\sequentially}{}

%\newcommand{\online}{}
\newcommand{\offline}{}

\newcommand{\private}{}
\newcommand{\semipublic}{}
%\newcommand{\public}{}
\newcommand{\stationary}{}
%\newcommand{\onthego}{}

%\newcommand{\leanback}{}
\newcommand{\leanforward}{}

\newcommand{\single}{}
\newcommand{\collaboration}{}
%\newcommand{\facetoface}{}
\newcommand{\sidetoside}{}
%\newcommand{\cornertocorner}{}

\newcommand{\smalltask}{}
%\newcommand{\repeatedtask}{}
%\newcommand{\locationbased}{}
\newcommand{\distraction}{}
%\newcommand{\urgent}{} 

\newcommand{\notvalidcontext}{1. Wenn die zu erweiternde Ansicht eine geringe Auflösung hat, verschlechtert sich die Qualität noch weiter, wenn sie auf einer größeren Bildschirmfläche angezeigt wird. 2. Die benutzten Bildschirme sollten sich von der Größe nicht zu sehr Unterscheiden, da sonst der Zuwachs an Fläche abnimmt(Siehe Abbildung).}


\newcommand{\devicetabular}{
\begin{tabular}[H]{|c|c|c|c|c|c|}
\hline 
\diagbox{von}{nach}   & Smartwatch & Smartphone & Tablet & Tabletop & Screens \\ 
\hline 
Smartwatch            &     x      &            &        &          &         \\ 
\hline 
Smartphone            &            &     x      &   x    &     (x)* &         \\ 
\hline 
Tablet                &            &     x      &   x    &     x    &         \\ 
\hline 
Tabletop              &            &     (x)*   &   x    &     x    &         \\ 
\hline
Screens               &            &            &        &          &         \\ 
\hline 
\end{tabular} 
\\
\\ *Nur bedingt nutzbar (Siehe Abzuratender Nutzungskontext, 2.)}


% -------------------------------
% WARUM
% -------------------------------

%\newcommand{\established}{}
\newcommand{\candidate}{}
\newcommand{\realizable}{}
%\newcommand{\futuristic}{}

\newcommand{\otherpatterns}{
\begin{itemize}
\item Stitch To Give
\item Stitch To Take
\item Stitch To Connect
\item Pinch To Connect
\item Spread To Disconnect
\end{itemize}
}

\newcommand{\stateoftheart}{
\begin{enumerate}
\item Erwähnung von Stitch-Gesten und Ableitung der Pinch-Geste \cite{Lucero2010}
\item Erste wissenschaftliche Erwähnung von Stitch-Gesten \cite{Hinckley2004}.
\end{enumerate}
}


\newcommand{\designprinciples}{}

\newcommand{\imageschemata}{}
%\newcommand{\imageSchemaContainer}{}
%\newcommand{\imageSchemaInOut}{}
\newcommand{\imageSchemaPath}{}
\newcommand{\imageSchemaSourcePathGoal}{}
%\newcommand{\imageSchemaUpDown}{}
%\newcommand{\imageSchemaLeftRight}{}
%\newcommand{\imageSchemaNearFar}{}
%\newcommand{\imageSchemaPartWhole}{}

\newcommand{\realworld}{}
\newcommand{\realworldNaivePhysic}{}
\newcommand{\realworldBodyAwareness}{}
\newcommand{\realworldEnvironmentAwareness}{}
\newcommand{\realworldSocialAwareness}{}

\newcommand{\metaphor}{}
\newcommand{\metaphordesc}{Das Zusammennähen von zwei Stoffteilen}

% -------------------------------
% TECHNISCHES
% -------------------------------


% -------------------------------
% SONSTIGES
% -------------------------------

\newcommand{\authors}{
Valentina Burjan, Hochschule Mannheim\\
Dominick Madden, Hochschule Mannheim\\
Horst Schneider, Hochschule Mannheim}
\newcommand{\versionhistory}{16.09.2016}
\newcommand{\dateofcreation}{17.09.2015}
\newcommand{\comments}{...}
\newcommand{\questions}{...}


% template inkludieren --------------

\begin{document}

% ------ fixes the build for all patterns where those new variables haven't been defined yet
\ifdefined\reactionSen
\else
\newcommand{\reactionSen}{tbd.}
\fi

\ifdefined\reactionRec
\else
\newcommand{\reactionRec}{tbd.}
\fi

\ifdefined\microinteractionstabular
\else
\newcommand{\microinteractionstabular}{tbd.}
\fi

\ifdefined\animations
\else
\newcommand{\animations}{tbd.}
\fi

\ifdefined\requiredTechnologies
\else
\newcommand{\requiredTechnologies}{tbd.}
\fi

\ifdefined\implementation
\else
\newcommand{\implementation}{tbd.}
\fi

\maketitle

%----------------------------
% CATEGORY ICON
%----------------------------
\begin{textblock}{2}[0,0](8, -3)
\ifthenelse{\equal{\category}{give}}{\newcommand{\icon}{icon_give.png}}{}
\ifthenelse{\equal{\category}{take}}{\newcommand{\icon}{icon_take.png}}{}
\ifthenelse{\equal{\category}{connect}}{\newcommand{\icon}{icon_connect.png}}{}
\ifthenelse{\equal{\category}{extend}}{\newcommand{\icon}{icon_extend.png}}{}
\ifthenelse{\equal{\category}{exchange}}{\newcommand{\icon}{icon_exchange.png}}{}	
\includegraphics[scale=0.5]{\icon}
\end{textblock}

% -------------------------------
% WAS
% -------------------------------
\section*{Was}

\subsection*{Problem}
\desc

\subsection*{Lösung}
\solution

\subsection*{Grafische Darstellung}
\begin{figure}[H]
\IfFileExists{\jobname_graphical_description.png}{\includegraphics[width=\textwidth]{\grafischedarstellung}}{}
\end{figure}

\subsection*{Kategorie}
\ifthenelse{\equal{\category}{give}}{$\boxtimes$}{$\Box$} Give   |   
\ifthenelse{\equal{\category}{take}}{$\boxtimes$}{$\Box$} Take   |   
\ifthenelse{\equal{\category}{exchange}}{$\boxtimes$}{$\Box$} Exchange   |   
\ifthenelse{\equal{\category}{extend}}{$\boxtimes$}{$\Box$} Extend   |   
\ifthenelse{\equal{\category}{connect}}{$\boxtimes$}{$\Box$} Connect

% -------------------------------
% WIE
% -------------------------------
\newpage
\section*{Wie}

\subsection*{Aktion des \sen s}
\useraction

\subsection*{Reaktionen des \sendev s}
\reactionSen

\subsection*{Reaktionen des \recdev s}
\reactionRec

\subsection*{Übersicht über die Microinteractions}
\microinteractionstabular

\subsubsection*{Animationen}
\animations

\subsection*{Hinweise zur Gestaltung der Interaktion}
\designnotes

% -------------------------------
% WANN
% -------------------------------

\section*{Wann}

\subsection*{Geeigneter Nutzungskontext}
\validcontext

\subsubsection*{Zeit}
\checkbox{\simultaneously} gleichzeitige Nutzung der beteiligten Geräte \\
\checkbox{\sequentially} sequentielle Nutzung der beteiligten Geräte

\subsubsection*{Ort}
\checkbox{\private} privat \\
\checkbox{\semipublic} halb-öffentlich \\
\checkbox{\public} öffentlich \\
\checkbox{\stationary} stationär \\
\checkbox{\onthego} unterwegs 

\subsubsection*{Körperhaltung der Benutzer}
\checkbox{\leanback} Lean-Back \\
\checkbox{\leanforward} Lean-Forward 

\subsubsection*{Teilnehmer}
\checkbox{\single} Einzelnutzer \\
\checkbox{\collaboration} Kollaboration

\subsubsection*{Anordnung zwischen Sender und Empfänger}
\checkbox{\facetoface} Face-To-Face \\
\checkbox{\sidetoside} Side-To-Side

\subsection*{Abzuratender Nutzungskontext}
\notvalidcontext

\subsection*{Geräteklassen}
\devicetabular


% -------------------------------
% WARUM
% -------------------------------

\section*{Warum}
\checkbox{\established} Bewährtes Interaction Pattern \\
\checkbox{\candidate} Interaction Pattern Kandidat: 
\checkbox{\realizable} realisierbar oder
\checkbox{\futuristic} futuristisch

\subsection*{Verwandte Patterns}
\otherpatterns

\subsection*{State of the Art}
\stateoftheart

\subsection*{Checkliste: Entspricht die Interaktion der Definiton einer "Blended Interaction"?}
\checkbox{\designprinciples} Werden die Designprinzipien berücksichtigt?
\begin{itemize}
\item[-] Die Interaktion greift eine Metapher aus der physikalischen Welt auf.
\item[-] Die Interaktion kann in einer Kollaboration ausgeführt werden.
\item[-] Die Interaktion unterstützt einen Workflow/eine Aufgabe.
\item[-] Die Interaktion findet in einer physikalischen Umgebung statt.
\end{itemize} 

\checkbox{\imageschemata} Image Schema/ta liegen zu Grunde.
\begin{itemize}
\item[-] \checkbox{\imageSchemaContainer} Container
\item[-] \checkbox{\imageSchemaInOut} In-Out
\item[-] \checkbox{\imageSchemaPath} Path
\item[-] \checkbox{\imageSchemaSourcePathGoal} Source-Path-Goal
\item[-] \checkbox{\imageSchemaUpDown} Up-Down
\item[-] \checkbox{\imageSchemaLeftRight} Left-Right
\item[-] \checkbox{\imageSchemaNearFar} Near-Far
\item[-] \checkbox{\imageSchemaPartWhole} Part-Whole
\end{itemize}

\checkbox{\realworld} Die real-weltlichen Kenntnisse des Menschen werden berücksichtigt.
\begin{itemize}
\item[-] \checkbox{\realworldNaivePhysic} Naive Physik
\item[-] \checkbox{\realworldBodyAwareness} Body Awareness and Skills
\item[-] \checkbox{\realworldEnvironmentAwareness} Environmental Awareness and Skills
\item[-] \checkbox{\realworldSocialAwareness} Social Awareness and Skills
\end{itemize}

\checkbox{\metaphor} Es ist eine natürliche Interaktion. Metapher/Assoziation: \metaphordesc

% -------------------------------
% TECHNISCHES
% -------------------------------

\section*{Technisches}

\subsection*{Benötigte Technologien}
\requiredTechnologies

\subsection*{Implementierungshinweise}
\implementation

% -------------------------------
% SONSTIGES
% -------------------------------

\section*{Sonstiges}

\subsection*{Autor/en}
\authors

\subsection*{Versionshistorie}
Erstelldatum: \dateofcreation \\
Letzte Änderung am: \versionhistory

\subsection*{Kommentare}
\comments

\subsection*{Offene Fragen}
\questions

\listoffigures

\printbibliography

\clearpage

\printglossaries

\end{document}