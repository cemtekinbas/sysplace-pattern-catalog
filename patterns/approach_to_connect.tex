\documentclass[11pt,a4paper,notitlepage]{article}

\usepackage[utf8]{inputenc}
\usepackage[T1]{fontenc}
\usepackage[german]{babel}
\usepackage{float}
%\usepackage{amsmath}
%\usepackage{amsfonts}
\usepackage{amssymb}
\usepackage{graphicx}
\usepackage{ifthen}
\usepackage{diagbox}
\usepackage[hyphens]{url}
\usepackage{textcomp,gensymb}
\usepackage{makecell}
\usepackage{textpos}
\usepackage{tabularx}
\usepackage{csquotes}
%\usepackage{hyperref}
\usepackage[natbib=true,bibstyle=numeric,backend=bibtexu,citestyle=numeric]{biblatex}
\bibliography{lit.bib} 
\renewcommand\theadalign{cb}
\renewcommand\theadfont{\bfseries}
\renewcommand\theadgape{\Gape[4pt]}

\usepackage{nopageno}

\author{}
\date{}
\title{\name}

% Template für Checkboxen ----
\newcommand{\checkbox}[1]{
\ifx#1\undefined
  $\Box$
\else
  $\boxtimes$  
\fi}

\setlength{\parindent}{0pt}

%----------------------------

\newcommand{\grafischedarstellung}{\jobname_graphical_description.png}

%----------------------------

\newcommand{\umldiagram}{\jobname_uml.png}

%----------------------------

\newcommand{\sequencediagram}{\jobname_sequence.png}

%----------------------------

\newcommand{\solutionimg}{\jobname_solution.png}

%----------------------------

\newcommand{\prototypeimg}{\jobname_prototype.png}

% --------- Glossary
\newcommand{\sen}{Sender}
\newcommand{\rec}{Empfänger}
\newcommand{\recdev}{Empfangsgerät}
\newcommand{\sendev}{Sendegerät}
\newcommand{\data}{Datenobjekt}

\newcommand{\name}{Approach To Connect}

% -------------------------------
% WAS
% -------------------------------

\newcommand{\desc}{Ein Benutzer möchte zur Datenübertragung eine Verbindung zwischen seinem Gerät und einem weiteren, in der Nähe befindlichen Gerät aufbauen.}

\newcommand{\solution}{Ein Benutzer nähert (\textit{approach}) sich mit seinem Gerät (dem \sendev{}) einem entfernten Gerät (dem \recdev{}). Nähert er sich dem \recdev{} bis auf einen bestimmten räumlichen Abstand, wird eine Verbindung hergestellt, die zur Datenübertragung genutzt werden kann.}

%\newcommand{\category}{give}
%\newcommand{\category}{take}
%\newcommand{\category}{exchange}
%\newcommand{\category}{extend}
\newcommand{\category}{connect}

% -------------------------------
% WIE
% -------------------------------

\newcommand{\useraction}{Der Benutzer hält ein mobiles \sendev{} in der Hand und nähert sich einem \recdev{} (z.B. einem großen Wanddisplay oder einem Tablet auf dem Tisch). Zwischen \sendev{} und \recdev{} besteht keine Verbindung. Wird nun ein vordefinierter Mindestabstand (Abstands-Schwellwert) unterschritten, initiiert eines der Geräte (üblicherweise das \sendev{}) den Verbindungsaufbau, indem es einen Verbindungsanfrage an das andere sendet.}

\newcommand{\reactionSen}{Besteht keine Verbindung zwischen den Geräten, ist das \sendev{} in einer Art Monitoring-Modus, in dem es auf Geräte in der Umgebung wartet, was dem Benutzer visuell durch statisches oder kontinuierliches visuelles Feedback sichtbar gemacht werden kann. Unterschreitet der Benutzer den Abstands-Schwellwert, erhält er auf dem \sendev{} ein Feedback über das erkannte \recdev{} und den Beginn des Verbindungsaufbaus (z.B. akustisch oder visuell). Bei \textit{erfolgreichem} oder \textit{gescheiterten} Verbindungsaufbau erhält der Benutzer ein entsprechendes Feedback, das wiederum z.B. aus einem akustischen Signal oder Vibration bestehen kann.}

\newcommand{\reactionRec}{Das Empfangsgerät hat eine eher passive Rolle und sollte dann Feedback geben, wenn ein mobiles Gerät erkannt wurde und ein Verbindungsaufbau stattfindet, sodass die Reaktion des \sendev{}s und des \recdev{}s gleichzeitig stattfinden und die Zuordnung der Geräte zueinander über das Feedback sichtbar wird.}

\newcommand{\microinteractionstabular}{\textbf{Dominick}}
\newcommand{\animations}{\textbf{Dominick}}

\newcommand{\designnotes}{Der Abstands-Schwellwert darf nicht zu niedrig definiert werden, da  man sonst das \sendev{} auf das \recdev{} auflegen müsste bzw. sehr nah vor einem großen Bildschirm stehen würde. Zudem sollte die Annäherung eindeutig sein, d.h. es sollten innerhalb eines Interaktionskontexts keine konkurrierenden Annäherungsgesten möglich sein.}

% -------------------------------
% WANN
% -------------------------------

\newcommand{\validcontext}{Verbinden von mobilen und stationären Geräten, Verbinden von Privatgeräten mit (halb-)öffentlichen Displays.}

\newcommand{\simultaneously}{}
%\newcommand{\sequentially}{}

\newcommand{\online}{}
\newcommand{\offline}{}

\newcommand{\private}{}
\newcommand{\semipublic}{}
%\newcommand{\public}{}
\newcommand{\stationary}{}
%\newcommand{\onthego}{}

%\newcommand{\leanback}{}
\newcommand{\leanforward}{}

\newcommand{\single}{}
\newcommand{\collaboration}{}
\newcommand{\facetoface}{}
%\newcommand{\sidetoside}{}
%\newcommand{\cornertocorner}{}

%\newcommand{\smalltask}{}
%\newcommand{\repeatedtask}{}
%\newcommand{\locationbased}{}
%\newcommand{\distraction}{}
%\newcommand{\urgent}{} 

\newcommand{\notvalidcontext}{Aufbauen einer Verbindung zu vertraulichen Geräten oder zum Austausch vertraulicher Daten; Sichtbarmachen vertraulicher Informationen (z.B. Name, Alter etc.) auf öffentlichen Displays.}


\newcommand{\devicetabular}{
\begin{tabular}[H]{|c|c|c|c|c|c|}
\hline 
\diagbox{von}{nach}   & Smartwatch & Smartphone & Tablet & Tabletop & Screens \\ 
\hline 
Smartwatch            &            &     x      &   x    &     x    &     x   \\ 
\hline 
Smartphone            &            &     x      &   x    &     x    &     x   \\ 
\hline 
Tablet                &            &            &   x    &     x    &     x   \\ 
\hline 
Tabletop              &            &            &        &          &         \\ 
\hline
Screens               &            &            &        &          &         \\ 
\hline 
\end{tabular} }

% -------------------------------
% WARUM
% -------------------------------

%\newcommand{\established}{}
\newcommand{\candidate}{}
\newcommand{\realizable}{}
%\newcommand{\futuristic}{}

\newcommand{\otherpatterns}{
\begin{itemize}
\item Approach To Give
\item Approach To Take
\item Approach To Extend
\end{itemize}
}

\newcommand{\stateoftheart}{
\begin{enumerate}
\item Aufbauen einer Bluetooth-Verbindung beim Betreten des Raums \cite{Dachselt2009}.
\item Kommunikation verschiedener Personen mit einem Fernseher basierend auf der Nähe \cite{Greenberg2011}.
\item Theoretische und praktische Ansätze zu \textit{Proxemics}, behandelt auch negative Nutzungskontexte (sog. Dark Patterns) \cite{Marquardt2015}.
\end{enumerate}
}


\newcommand{\designprinciples}{}

\newcommand{\imageschemata}{}
%\newcommand{\imageSchemaVoid}{}
%\newcommand{\imageSchemaObject}{}
%\newcommand{\imageSchemaSubstance}{}
\newcommand{\imageSchemaCenterPeriphery}{}
%\newcommand{\imageSchemaContact}{}
%\newcommand{\imageSchemaFrontBack}{}
%\newcommand{\imageSchemaLocation}{}
\newcommand{\imageSchemaNearFar}{}
\newcommand{\imageSchemaPath}{}
%\newcommand{\imageSchemaSourcePathGoal}{}
%\newcommand{\imageSchemaScale}{}
%\newcommand{\imageSchemaLeftRight}{}
%\newcommand{\imageSchemaContainer}{}
%\newcommand{\imageSchemaContent}{}
%\newcommand{\imageSchemaFullEmpty}{}
\newcommand{\imageSchemaInOut}{}
%\newcommand{\imageSchemaSurface}{}
%\newcommand{\imageSchemaMerging}{}
%\newcommand{\imageSchemaSplitting}{}
%\newcommand{\imageSchemaMomentum}{}
%\newcommand{\imageSchemaSelfMotion}{}
%\newcommand{\imageSchemaBigSmall}{}
%\newcommand{\imageSchemaFastSlow}{}
%\newcommand{\imageSchemaPartWhole}{}

\newcommand{\realworld}{}
\newcommand{\realworldNaivePhysic}{}
\newcommand{\realworldBodyAwareness}{}
\newcommand{\realworldEnvironmentAwareness}{}
\newcommand{\realworldSocialAwareness}{}

\newcommand{\metaphor}{}
\newcommand{\metaphordesc}{Magnete (die sich bei geringer Distanz verbinden durch ihre Pole)}

% -------------------------------
% TECHNISCHES
% -------------------------------

\newcommand{\requiredTechnologies}{
\begin{itemize}
\item Liste benötigter Technologien
\item Input/Output/Connectivity
\item Grafik
\end{itemize}
}
\newcommand{\implementation}{
\begin{itemize}
\item genereller Ablauf der Geste (Grafik + Erklärung)
\item Erkennung der Geste (Grafik + Erklärung)
\item Constraint Check (Grafik + Erklärung)
\item Anknüpfpunkt zu Transfer // Überleitung Developer Page
\end{itemize}

\subsubsection*{Ablauf Gestenerkennung}
\subsubsection*{Approach Erkennung}
\subsubsection*{Approach Constraint Check}
}

% -------------------------------
% SONSTIGES
% -------------------------------

\newcommand{\authors}{Horst Schneider, Hochschule Mannheim\\
Dominick Madden, Hochschule Mannheim\\
Valentina Burjan, Hochschule Mannheim}
\newcommand{\literature}{...}
\newcommand{\figures}{...}
\newcommand{\versionhistory}{11.08.2016}
\newcommand{\dateofcreation}{15.08.2015}
\newcommand{\comments}{...}
\newcommand{\questions}{...}


% template inkludieren --------------

\begin{document}

% ------ fixes the build for all patterns where those new variables haven't been defined yet
\ifdefined\reactionSen
\else
\newcommand{\reactionSen}{tbd.}
\fi

\ifdefined\reactionRec
\else
\newcommand{\reactionRec}{tbd.}
\fi

\ifdefined\microinteractionstabular
\else
\newcommand{\microinteractionstabular}{tbd.}
\fi

\ifdefined\animations
\else
\newcommand{\animations}{tbd.}
\fi

\ifdefined\requiredTechnologies
\else
\newcommand{\requiredTechnologies}{tbd.}
\fi

\ifdefined\implementation
\else
\newcommand{\implementation}{tbd.}
\fi

\maketitle

%----------------------------
% CATEGORY ICON
%----------------------------
\begin{textblock}{2}[0,0](8, -3)
\ifthenelse{\equal{\category}{give}}{\newcommand{\icon}{icon_give.png}}{}
\ifthenelse{\equal{\category}{take}}{\newcommand{\icon}{icon_take.png}}{}
\ifthenelse{\equal{\category}{connect}}{\newcommand{\icon}{icon_connect.png}}{}
\ifthenelse{\equal{\category}{extend}}{\newcommand{\icon}{icon_extend.png}}{}
\ifthenelse{\equal{\category}{exchange}}{\newcommand{\icon}{icon_exchange.png}}{}	
\includegraphics[scale=0.5]{\icon}
\end{textblock}

% -------------------------------
% WAS
% -------------------------------
\section*{Was}

\subsection*{Problem}
\desc

\subsection*{Lösung}
\solution

\subsection*{Grafische Darstellung}
\begin{figure}[H]
\IfFileExists{\jobname_graphical_description.png}{\includegraphics[width=\textwidth]{\grafischedarstellung}}{}
\end{figure}

\subsection*{Kategorie}
\ifthenelse{\equal{\category}{give}}{$\boxtimes$}{$\Box$} Give   |   
\ifthenelse{\equal{\category}{take}}{$\boxtimes$}{$\Box$} Take   |   
\ifthenelse{\equal{\category}{exchange}}{$\boxtimes$}{$\Box$} Exchange   |   
\ifthenelse{\equal{\category}{extend}}{$\boxtimes$}{$\Box$} Extend   |   
\ifthenelse{\equal{\category}{connect}}{$\boxtimes$}{$\Box$} Connect

% -------------------------------
% WIE
% -------------------------------
\newpage
\section*{Wie}

\subsection*{Aktion des \sen s}
\useraction

\subsection*{Reaktionen des \sendev s}
\reactionSen

\subsection*{Reaktionen des \recdev s}
\reactionRec

\subsection*{Übersicht über die Microinteractions}
\microinteractionstabular

\subsubsection*{Animationen}
\animations

\subsection*{Hinweise zur Gestaltung der Interaktion}
\designnotes

% -------------------------------
% WANN
% -------------------------------

\section*{Wann}

\subsection*{Geeigneter Nutzungskontext}
\validcontext

\subsubsection*{Zeit}
\checkbox{\simultaneously} gleichzeitige Nutzung der beteiligten Geräte \\
\checkbox{\sequentially} sequentielle Nutzung der beteiligten Geräte

\subsubsection*{Ort}
\checkbox{\private} privat \\
\checkbox{\semipublic} halb-öffentlich \\
\checkbox{\public} öffentlich \\
\checkbox{\stationary} stationär \\
\checkbox{\onthego} unterwegs 

\subsubsection*{Körperhaltung der Benutzer}
\checkbox{\leanback} Lean-Back \\
\checkbox{\leanforward} Lean-Forward 

\subsubsection*{Teilnehmer}
\checkbox{\single} Einzelnutzer \\
\checkbox{\collaboration} Kollaboration

\subsubsection*{Anordnung zwischen Sender und Empfänger}
\checkbox{\facetoface} Face-To-Face \\
\checkbox{\sidetoside} Side-To-Side

\subsection*{Abzuratender Nutzungskontext}
\notvalidcontext

\subsection*{Geräteklassen}
\devicetabular


% -------------------------------
% WARUM
% -------------------------------

\section*{Warum}
\checkbox{\established} Bewährtes Interaction Pattern \\
\checkbox{\candidate} Interaction Pattern Kandidat: 
\checkbox{\realizable} realisierbar oder
\checkbox{\futuristic} futuristisch

\subsection*{Verwandte Patterns}
\otherpatterns

\subsection*{State of the Art}
\stateoftheart

\subsection*{Checkliste: Entspricht die Interaktion der Definiton einer "Blended Interaction"?}
\checkbox{\designprinciples} Werden die Designprinzipien berücksichtigt?
\begin{itemize}
\item[-] Die Interaktion greift eine Metapher aus der physikalischen Welt auf.
\item[-] Die Interaktion kann in einer Kollaboration ausgeführt werden.
\item[-] Die Interaktion unterstützt einen Workflow/eine Aufgabe.
\item[-] Die Interaktion findet in einer physikalischen Umgebung statt.
\end{itemize} 

\checkbox{\imageschemata} Image Schema/ta liegen zu Grunde.
\begin{itemize}
\item[-] \checkbox{\imageSchemaContainer} Container
\item[-] \checkbox{\imageSchemaInOut} In-Out
\item[-] \checkbox{\imageSchemaPath} Path
\item[-] \checkbox{\imageSchemaSourcePathGoal} Source-Path-Goal
\item[-] \checkbox{\imageSchemaUpDown} Up-Down
\item[-] \checkbox{\imageSchemaLeftRight} Left-Right
\item[-] \checkbox{\imageSchemaNearFar} Near-Far
\item[-] \checkbox{\imageSchemaPartWhole} Part-Whole
\end{itemize}

\checkbox{\realworld} Die real-weltlichen Kenntnisse des Menschen werden berücksichtigt.
\begin{itemize}
\item[-] \checkbox{\realworldNaivePhysic} Naive Physik
\item[-] \checkbox{\realworldBodyAwareness} Body Awareness and Skills
\item[-] \checkbox{\realworldEnvironmentAwareness} Environmental Awareness and Skills
\item[-] \checkbox{\realworldSocialAwareness} Social Awareness and Skills
\end{itemize}

\checkbox{\metaphor} Es ist eine natürliche Interaktion. Metapher/Assoziation: \metaphordesc

% -------------------------------
% TECHNISCHES
% -------------------------------

\section*{Technisches}

\subsection*{Benötigte Technologien}
\requiredTechnologies

\subsection*{Implementierungshinweise}
\implementation

% -------------------------------
% SONSTIGES
% -------------------------------

\section*{Sonstiges}

\subsection*{Autor/en}
\authors

\subsection*{Versionshistorie}
Erstelldatum: \dateofcreation \\
Letzte Änderung am: \versionhistory

\subsection*{Kommentare}
\comments

\subsection*{Offene Fragen}
\questions

\listoffigures

\printbibliography

\clearpage

\printglossaries

\end{document}